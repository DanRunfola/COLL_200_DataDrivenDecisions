% Latex Document modified from Latex design of Brian R. Hall (available at www.brianrhall.net)

% Document settings
\documentclass[11pt]{article}
\usepackage[margin=1in]{geometry}
\usepackage[pdftex]{graphicx}
\usepackage{multirow}
\usepackage{setspace}
\pagestyle{plain}
\usepackage{color}
\setlength\parindent{0pt}
\makeatletter
\setlength{\@fptop}{0pt}
\makeatother

\begin{document}

% Course information
  \begin{center}

{\includegraphics[height=1.25in,width=1in]{wmchiffre1.jpg}} 

\LARGE Data Driven Decisionmaking (Fall 2017, APSC 490 (COLL 200))\\ \vspace{3mm}
\end{center}
\large \textbf{Schedule} \\
\normalsize Tuesday and Thursday 9:30AM to 10:50AM \textit{Phi Beta Kappa Hall 221} \\
\vspace{2mm}

\begin{table}[ht]
\begin{tabular}{l l}
\large \textbf{Instructor} & \\
\large Dr. \textbf{Dan} Runfola \\
\large danr@wm.edu   \\
\large ISC 1269 \\
\large Office Hours: 11-1P R \\
\large 757.221.1970  \\
\end{tabular}
\end{table}


\textit{Please let me know if you have any documented disabilities that may impact your performance in this class.}

\textbf {\large \\ Course Description:} For the increasing volume of information being produced to be useful in decision-making processes, it needs to be systematically organized and analyzed. This course will provide students with an opportunity to apply quantitative methods to a wide variety of real-world problems defined by decision makers from federal and international policy making groups. Course work will include applying a wide set of techniques (such as the analytic hierarchy process, ordered weighting averaging) which integrate human preferences and perception with quantitative information, with an emphasis on uncertainty. Students will learn about and consider the challenges associated with data reduction – how to balance between the limits of human perception, the value of additional information, and temporal constraints imposed by the decision making process.  \\
\\
\textbf {Prerequisite(s):} None.

\textbf {Credit Hours:} 3 \\

\textbf {\large Materials:}\\ 
A free SciClone account is needed in order to access the William and Mary High Performance Cluster (HPC).  Registration can be started at https://hpc.wm.edu/acctreq/ . \\
\\
You must bring a laptop to class  each day unless otherwise noted during lecture.


\vspace{8mm}

\textbf {\large Course Objectives:} 
\begin{enumerate} \itemsep-0.4em
  \item Provide students with a critical understanding of the decision making process and the use of data and intuition on short deadlines.
  \item Develop student’s ability to communicate findings, analysis, and visualization skills for future courses (and jobs).
  \item Expose students to real-world problems that are being engaged with by contemporary problem solvers and decision makers.
  \item Provide an opportunity to earn credit towards the COLL 200 requirement in the Culture, Society, and the Individual domain (passing grade required).
\end{enumerate}
\vspace{8mm}

\textbf {\large Grade Distribution:} \\
\hspace*{40mm}
\begin{tabular}{ l l }
Reading Assignments & 10\% \\
Projects & 60\% \\
Final Report & 30\%\\
\end{tabular} \\\\

\textbf {\large Letter Grade Distribution:} \\
\hspace*{40mm}
\begin{tabular}{ l l | l l }
\textgreater= 93.00 & A & 73.00 - 76.99 & C \\
90.00 - 92.99 & A-  & 70.00 - 72.99 & C- \\
87.00 - 89.99 & B+  & 67.00 - 69.99 & D+ \\
83.00 - 86.99 & B  & 63.00 - 66.99 & D \\
80.00 - 82.99 & B-  & 60.00 - 62.99 & D- \\
77.00 - 79.99 & C+  & \textless= 59.99 & F \\
\end{tabular} \\
.\\

\textbf {\large Time Commitment:} Excelling in college level course work typically requires on average three to four hours per credit per week.  Since this is a three credit course, in addition to the time we meet as a class each week, you should expect to spend nine to twelve hours on average reading, writing, or otherwise preparing for this class on a weekly basis.\\

\textbf {\large Attendance:} This class does not have an attendance policy.  However, it will be difficult to learn enough to pass the class without regular participation, as the majority of course content relevant to tests and assignments will be covered in class.\\

\textbf {\large Discussions and Reading Assignments:} Most lecture sessions will begin with a discussion of the assigned materials.  As such, most weeks students are required to write a short summary (no more than 1 page) reflecting on a given weeks assigned readings - these summaries can represent questions the material raised, commentary, or critiques. \\

\textbf {\large Classroom Behavior:} Please remain civil during discussions to promote the open exchange of ideas and foster a culture of open dialogue.  Please bear in mind that all students are entitled to their own opinion.  You are expected to listen attentively to each person speaking.  Please refrain from eating during class (and, if you must, make sure it isn't loud!).\\

\textbf {\large Teacher-student conferences:} Students performing at a C level or below are required to schedule a meeting with the instructor to discuss class performance.\\

\textbf {\large Late / Poor Performance Policy:} Assignments will not be accepted late, excepting in documented circumstances (i.e., an illness with a doctor's note). \\

\textbf {\large Final Project:} The final report will involve the application of the skills taught throughout the semester, and will represent your ability to provide an answer to a real-world question using the data and skills at your disposal. It will include a 5 page written report (inclusive of data visualizations and tables) designed as an executive summary for a decisionmaker.\\

\textbf {\large Important Dates:} The add and drop deadline this semester is September 8th, and withdrawal deadline is October 27th.

\vspace{4mm}
\textbf {\LARGE Do not cheat!} \\
.\hrulefill . \\
\textbf{Academic dishonesty is taken very seriously.  Make sure to cite all of your work, and do not turn in work that is not yours!  Cases of academic dishonesty will be evaluated and acted upon in accordance with William and Mary policies, which can be found at http://www.wm.edu/offices/deanofstudents/services/
studentconduct/} \\
.\hrulefill . \\
\vspace{8mm}

% Course Outline
\textbf {\large Course Outline}:

The course outline can be found below.  The weekly content might change as it depends on the progress of the class.

\begin{table}[h!]
\small % The size of the table text can be changed depending on content. Remove if desired.
\begin{tabular}{ | c | c | }
\hline
\textbf{Week} & \textbf{Content} \\
\hline
Week 1 & \begin{minipage}{.85\textwidth}
\begin{itemize} \itemsep-0.4em
	\vspace{1mm}
	\item 8/31 - Introduction
	\vspace{1mm}
\end{itemize}
\end{minipage} \\
\hline

Week 2 & \begin{minipage}{.85\textwidth}
\begin{itemize} \itemsep-0.4em
	\vspace{1mm}
	\item 9/5 - Eric Walter, HPC Account Setup; 9/7 - Interactive Learning 
	\vspace{1mm}
\end{itemize}
\end{minipage} \\
\hline

Week 3 & \begin{minipage}{.85\textwidth}
\begin{itemize} \itemsep-0.4em
	\vspace{1mm}
	\item 9/12 and 9/14 - Asking the Right Questions
	
	\item Reading (read before 9/12): Critical Questions for Big Data
	\item Reading (read before 9/14): Algorithmic Accountability
	
	\item Project 1 assigned, due midnight Fri Sept 22 - Defining and Simulating Uncertainty. 
	\vspace{1mm}
\end{itemize}
\end{minipage} \\
\hline

Week 4 & \begin{minipage}{.85\textwidth}
\begin{itemize} \itemsep-0.4em
	\vspace{1mm}
	\item 9/18 and 9/21 - Non-traditional Data Sources
	
	\item Reading (read before 9/18): "One Vast Index": Google Book Search 
	\item Reading (read before 9/21): Content Analysis in an Era of Big Data
		
	\item Project 2 assigned, due midnight Fri Oct 6 -  Retrieving Data from Non-traditional Sources
	\vspace{1mm}
\end{itemize}
\end{minipage} \\
\hline

Week 5 & \begin{minipage}{.85\textwidth}
\begin{itemize} \itemsep-0.4em
	\vspace{1mm}
	\item 9/26 and 9/28 - Picking the Right Data 
	
	\item Reading (read before 9/26): The parable of Google Flu
	\item Reading (read before 9/28): Salmon and Red Herrings
	

	\item Reading Assignment 1 due by Midnight Friday, September 29th.

	\vspace{1mm}
\end{itemize}
\end{minipage} \\
\hline


Week 6 & \begin{minipage}{.85\textwidth}
\begin{itemize} \itemsep-0.4em
	\vspace{1mm}
	\item 10/3 and 10/5 - Picking the Right Models

	\item Reading (read before 10/3): An Introduction to Model Selection
	\item Reading (read before 10/5): Model Selection: An Integral Part of Inference
	
	\item Project 2 due midnight, Friday October 6th.
	
	
	\vspace{1mm}
\end{itemize}
\end{minipage} \\
\hline

Week 7 & \begin{minipage}{.85\textwidth}
\begin{itemize} \itemsep-0.4em
	\vspace{1mm}
	\item 10/10 and 10/12 - Distributional Assumptions

	\item Reading (Read before 10/10): Same Stats, Different Graphs 
	\item Reading (Read before 10/12): The Insignificance of Statistical Significane Testing
	
		\item Project 3 assigned, due midnight Monday October 23th - Integrating Disparate Data Sources
	
	
	\vspace{1mm}
\end{itemize}
\end{minipage} \\
\hline

Week 8 & \begin{minipage}{.85\textwidth}
\begin{itemize} \itemsep-0.4em
	\vspace{1mm}
	\item 10/17 - No Class (Fall Break)
	\item 10/19 - Data Visualization (No Reading)
	
	\item Project 3 due by midnight, Monday October 23.
	\item Project 4 assigned, due midnight Friday 11/3 - Simulating a Solution
	\vspace{1mm}
\end{itemize}
\end{minipage} \\
\hline

\end{tabular} 
\end{table}

\pagebreak

\begin{table}[h!]
\small % The size of the table text can be changed depending on content. Remove if desired.
\begin{tabular}{ | c | c | }
\hline
\textbf{Week} & \textbf{Content} \\
\hline

Week 9 & \begin{minipage}{.85\textwidth}
\begin{itemize} \itemsep-0.4em
	\vspace{1mm}
	\item 10/24 and 10/26 - Quantifying Uncertainty
	
	\item Reading (Read before 10/24): Survey of Sampling-Based Methods for Uncertainty and Sensitivity Analysis pg 1-12.
	\item Reading (Read before 10/26): Survey of Sampling-Based Methods for Uncertainty and Sensitivity Analysis pg 13 - 24.
	
	\item Reading Assignment 2 due by Midnight Friday, October 27th
	
	\vspace{1mm}
\end{itemize}
\end{minipage} \\
\hline

Week 10 & \begin{minipage}{.85\textwidth}
\begin{itemize} \itemsep-0.4em
	\vspace{1mm}
	\item 10/31 and 11/2 - Human Perception and Data Manipulation
	
	\item Reading (Read before 10/31): Belief in the Law of Small Numbers
	\item Reading (Read before 11/2): Visual Quality Metrics and Human Perception
	
	
	\item Project 4 due by Midnight Friday, November 3rd.
	\item Project 5 assigned, due midnight Friday 12/1 - Solution Set Optimization with Decisionmaker Input
	
	\vspace{1mm}
\end{itemize}
\end{minipage} \\
\hline

Week 11 & \begin{minipage}{.85\textwidth}
\begin{itemize} \itemsep-0.4em
	\vspace{1mm}
	\item 11/7 and 11/9 - Human Perception in Decision and Policymaking	

	\item Reading (Read before 11/7): Known Knowns, Known Unknowns, Unknown Unkowns: The Predicament of Evidence-Based Policy
	\item Reading (Read before 11/9):The Monty Hall Problem
	
	\item Reading Assignment 3 due by Midnight Friday, November 10th.	
	
	\vspace{1mm}
\end{itemize}
\end{minipage} \\
\hline

Week 12 & \begin{minipage}{.85\textwidth}
\begin{itemize} \itemsep-0.4em
	\vspace{1mm}
	\item 11/14 and 11/16 - Nonparametric Modeling in Decision Making
	
	\item Reading (Read before 11/14): Of Prediction and Policy
	\item Reading (Read before 11/16): Dealing with Nonnormal Data
	
	\item Reading Assignment 4 due by Midnight Friday, November 17th.	
	
	\vspace{1mm}
\end{itemize}
\end{minipage} \\
\hline

Week 13 & \begin{minipage}{.85\textwidth}
\begin{itemize} \itemsep-0.4em
	\vspace{1mm}
	\item 11/21 - Hands on Learning, 11/23 - No class (Thanksgiving Break)
	\vspace{1mm}
\end{itemize}
\end{minipage} \\
\hline

Week 14 & \begin{minipage}{.85\textwidth}
\begin{itemize} \itemsep-0.4em
	\vspace{1mm}
	\item 11/27 and 11/30 - Communicating with Data

	\item Reading (Read before 11/27):Distilling Meaning from Data
	\item Reading (Read before 11/30): Social Influence of Big Data (a 61-million person experiment)
	
	\item Project 5 due on Friday, December 1st by midnight.	
	
	\vspace{1mm}
\end{itemize}
\end{minipage} \\
\hline

Week 15 & \begin{minipage}{.85\textwidth}
\begin{itemize} \itemsep-0.4em
	\vspace{1mm}
	\item 12/5 and 12/7 - Trust, Data, Decisions and Society
	
	\item No reading assignment due this week.
	
	\item Final project assigned, due on December 17th by midnight.
	\vspace{1mm}
\end{itemize}
\end{minipage} \\
\hline

Final Project & \begin{minipage}{.85\textwidth}
\begin{itemize} \itemsep-0.4em
	\vspace{1mm}
	\item Due on December 17th by Midnight
	\vspace{1mm}
\end{itemize}
\end{minipage} \\
\hline

\end{tabular} 
\end{table}

\end{document}



